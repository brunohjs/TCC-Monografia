% RESUMO--------------------------------------------------------------------------------

\begin{resumo}[RESUMO]
\begin{SingleSpacing}

% Não altere esta seção do texto--------------------------------------------------------
\imprimirautorcitacao. \imprimirtitulo. \imprimirdata. \pageref {LastPage} f. \imprimirprojeto\ – \imprimirprograma, \imprimirinstituicao. \imprimirlocal, \imprimirdata.\\
%---------------------------------------------------------------------------------------

A exploração robótica de locais submersos ainda é um desafio para a humanidade, contudo o desenvolvimento de novas tecnologias e pesquisas científicas contribuem para o avanço da área.  
O sonar se torna uma ótima alternativa para estudar ambientes submersos, visto que ele dispensa a dependência de luminosidade no ambiente e ignora a turbidez da água. 
A utilização do sensor em pesquisas não é tão acessível por conta do alto custo de equipamentos e de expedições.
Contudo, a ideia do trabalho é utilizar um sonar de baixo custo e simuladores virtuais para realizar experimentos.
O trabalho propõe utilizar uma metodologia para estimar variações espaciais de superfícies submersas através da  reconstrução tridimensional, a partir de dados coletados com sonar MSIS (do inglês \textit{Mechanical Scanning Image Sonar}). 
A metodologia abordará um experimento em simulação virtual e um real, fazendo a comparação entre sí dos resultados obtidos.
O trabalho poderá servir como base para ser aplicado nas superfícies submersas de geleiras com o propósito de obter uma melhor mensuração da ablação das mesmas.


\vspace{2em}
\textbf{Palavras-chave}: Robótica Subaquática. Reconstrução 3D. \textit{Point Cloud}. Sonar MSIS.

\end{SingleSpacing}
\end{resumo}

% OBSERVAÇÕES---------------------------------------------------------------------------
% Altere o texto inserindo o Resumo do seu trabalho.
% Escolha de 3 a 5 palavras ou termos que descrevam bem o seu trabalho 
