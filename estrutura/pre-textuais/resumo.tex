% RESUMO--------------------------------------------------------------------------------

\begin{resumo}[RESUMO]
\begin{SingleSpacing}

% Não altere esta seção do texto--------------------------------------------------------
\imprimirautorcitacao. \imprimirtitulo. \imprimirdata. \pageref {LastPage} f. \imprimirprojeto\ – \imprimirprograma, \imprimirinstituicao. \imprimirlocal, \imprimirdata.\\
%---------------------------------------------------------------------------------------

O aquecimento global é o processo de aumento da temperatura média da superfície terrestre e dos oceanos causado por uma grande quantidade de gases originados de atividades humanas e emitidos na atmosfera terrestre. Um dos problemas ocasionado pelo aumento da temperatura é a diminuição da área de geleiras. Devido a importância econômica, ao seu papel como indicador de mudança no clima, além de ser o maior reservatório de água doce sobre a Terra, o monitoramento das geleiras se torna necessário e indispensável para o planeta. Esse trabalho propõe uma metodologia através do uso de algoritmos para estimar variações espaciais de superfícies submersas através da  reconstrução em 3D de superfícies a partir de dados coletados com sensor sonar MSIS. O trabalho servirá como base para ser aplicado nas superfícies submersas de geleiras com o propósito de obter uma melhor mensuração da ablação dessas superfícies.



\vspace{2em}
\textbf{Palavras-chave}: Robótica Subaquática. Reconstrução 3D. \textit{Point Cloud}, Sonar MSIS.

\end{SingleSpacing}
\end{resumo}

% OBSERVAÇÕES---------------------------------------------------------------------------
% Altere o texto inserindo o Resumo do seu trabalho.
% Escolha de 3 a 5 palavras ou termos que descrevam bem o seu trabalho 
