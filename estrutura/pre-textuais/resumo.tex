% RESUMO--------------------------------------------------------------------------------

\begin{resumo}[RESUMO]
\begin{SingleSpacing}

% Não altere esta seção do texto--------------------------------------------------------
\imprimirautorcitacao. \imprimirtitulo. \imprimirdata. \pageref {LastPage} f. \imprimirprojeto\ – \imprimirprograma, \imprimirinstituicao. \imprimirlocal, \imprimirdata.\\
%---------------------------------------------------------------------------------------

Imagens acústicas são grandes fontes de dados que podem ser exploradas de diferentes formas para diversos propósitos.
Um dos seus propósitos é a reconstrução em 3D com dados provenientes de imagens desse tipo, que é um ramo da ciência pouco estudado no mundo por conta da restrita aplicação e pelo alto custo dos equipamentos.
Porém, a área possui um grande potencial visto que, através dela é possível fazer a reconstrução de ambientes e locais submersos, os quais são pouco explorados e praticamente inacessíveis para o ser humano.
Além do mais, diversos trabalhos podem ser realizados com a reconstrução, como a estimação de volume e superfície em um determinado ambiente ou local submerso.

O trabalho propõe utilizar uma metodologia através do auxílio de algoritmos para estimar variações espaciais de superfícies submersas através da  reconstrução em 3D, a partir de dados coletados com sensor sonar MSIS (do inglês \textit{Mechanical Scanning Image Sonar}). 
O objetivo da utilização do modelo MSIS é a redução dos custos, visto que o risco de perda do equipamento é alto em locais onde há pouco contato com o operador.
O trabalho poderá servir como base para ser aplicado nas superfícies submersas de geleiras com o propósito de obter uma melhor mensuração da ablação das mesmas.


\vspace{2em}
\textbf{Palavras-chave}: Robótica Subaquática. Reconstrução 3D. \textit{Point Cloud}. Sonar MSIS.

\end{SingleSpacing}
\end{resumo}

% OBSERVAÇÕES---------------------------------------------------------------------------
% Altere o texto inserindo o Resumo do seu trabalho.
% Escolha de 3 a 5 palavras ou termos que descrevam bem o seu trabalho 
