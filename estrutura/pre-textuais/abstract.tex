% ABSTRACT--------------------------------------------------------------------------------

\begin{resumo}[ABSTRACT]
\begin{SingleSpacing}

% Não altere esta seção do texto--------------------------------------------------------
\imprimirautorcitacao. \imprimirtitleabstract. \imprimirdata. \pageref {LastPage} f. \imprimirprojeto\ – \imprimirprograma, \imprimirinstituicao. \imprimirlocal, \imprimirdata.\\
%---------------------------------------------------------------------------------------

%Elemento obrigatório em tese, dissertação, monografia e TCC. É a versão do resumo em português para o idioma de divulgação internacional. Deve ser antecedido pela referência do estudo. Deve aparecer em folha distinta do resumo em língua portuguesa e seguido das palavras representativas do conteúdo do estudo, isto é, das palavras-chave. Sugere-se a elaboração do resumo (Abstract) e das palavras-chave (Keywords) em inglês; para resumos em outras línguas, que não o inglês, consultar o departamento / curso de origem.\\

The robotic exploration of underwater locations is still a challenge for humanity, but the development of new technologies and scientific research contribute to the area advancement.
The sonar becomes a great alternative to studying submerged environments, since it dispenses with the dependence of environment luminosity and ignores the water turbidity.
The use of the sensor in research is not so accessible due to the high cost of equipment and shipments.
However, the idea of the work is to use low-cost sonar and virtual simulators to perform experiments.
The work proposes to use a methodology to estimate spatial variations of underwater surfaces through three-dimensional reconstruction, from data collected with MSIS sonar.
The methodology will approach an experiment in virtual simulation and a real simulation, comparing the obtained results.
The work may serve as a base to be applied to the glaciers underwater surfaces with the purpose to obtain a better glaciers ablation measurement.

\vspace{2em}
\textbf{Keywords}: Underwater Robotics. 3D Reconstruction. Point Cloud. Sonar MSIS.

\end{SingleSpacing}
\end{resumo}

% OBSERVAÇÕES---------------------------------------------------------------------------
% Altere o texto inserindo o Abstract do seu trabalho.
% Escolha de 3 a 5 palavras ou termos que descrevam bem o seu trabalho 
