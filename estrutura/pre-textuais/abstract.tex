% ABSTRACT--------------------------------------------------------------------------------

\begin{resumo}[ABSTRACT]
\begin{SingleSpacing}

% Não altere esta seção do texto--------------------------------------------------------
\imprimirautorcitacao. \imprimirtitleabstract. \imprimirdata. \pageref {LastPage} f. \imprimirprojeto\ – \imprimirprograma, \imprimirinstituicao. \imprimirlocal, \imprimirdata.\\
%---------------------------------------------------------------------------------------

%Elemento obrigatório em tese, dissertação, monografia e TCC. É a versão do resumo em português para o idioma de divulgação internacional. Deve ser antecedido pela referência do estudo. Deve aparecer em folha distinta do resumo em língua portuguesa e seguido das palavras representativas do conteúdo do estudo, isto é, das palavras-chave. Sugere-se a elaboração do resumo (Abstract) e das palavras-chave (Keywords) em inglês; para resumos em outras línguas, que não o inglês, consultar o departamento / curso de origem.\\

Acoustic images are great sources of data that can be explored in different ways for various purposes.
One of its purposes is the 3D reconstruction with data coming from images of this type, which is a branch of science understudied in the world due to the restricted application and the high cost of the equipment.
However, the area has a great potential since, through it, it is possible to reconstruct submerged environments and places, which are underexplored and practically inaccessible to humans.
Moreover, several works can be performed with reconstruction, such as volume and surface estimation in a given submerged environment or location.

The work proposes to use a methodology through the aid of algorithms to estimate spatial variations of submerged surfaces through 3D reconstruction, from data collected with MSIS (Mechanical Scanning Image Sonar) sonar sensor.
The objective of using the MSIS model is to cost reduce, since the risk of loss of equipment is high in places where there is low contact with the operator.
The work may serve as a base to be applied to the glaciers submerged surfaces in order to obtain a better measurement of the ablation of then.

\vspace{2em}
\textbf{Keywords}: Underwater Robotics. 3D Reconstruction. Point Cloud. Sonar MSIS.

\end{SingleSpacing}
\end{resumo}

% OBSERVAÇÕES---------------------------------------------------------------------------
% Altere o texto inserindo o Abstract do seu trabalho.
% Escolha de 3 a 5 palavras ou termos que descrevam bem o seu trabalho 
