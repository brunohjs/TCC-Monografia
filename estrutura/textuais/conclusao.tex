% CONCLUSÃO--------------------------------------------------------------------

\chapter{CONCLUSÃO}
\label{chap:conclusao}

Este capítulo apresenta a conclusão final do autor sobre o trabalho desenvolvido.
No Capítulo \ref{chap:fundamentacaoTeorica} foi apresentado o conteúdo sobre o sensor sonar e os \textit{softwares} utilizados, que são necessários para compreender o projeto.
O Capítulo \ref{chap:metodologia} mostrou como as ferramentas e conhecimento apresentados no capítulo anterior interagiram entre si, para formar a metodologia de trabalho.
As reconstruções e os resultados obtidos foram apresentados no Capítulo \ref{chap:resultados}.


\section{Considerações Finais}
\label{sec:consideracoes_finais}

Observa-se que nenhum resultado obteve 100\% na taxa de acerto, porque isso é praticamente uma utopia, entretanto sempre é possível melhorar.
Na reconstrução existe uma série de pequenos fatores que acabam diminuindo a taxa de acerto.
Por exemplo, alguns dos filtros complexos podem acabar deslocando levemente os pontos da nuvem de pontos, interferindo no momento do cálculo.
Do mesmo modo, existe a influência no momento da conversão das coordenadas polares para cartesianas do conjunto de pontos.

Contudo, os resultados foram satisfatórios e acredito que o trabalho pode ser sequenciado e ser aplicado em um ambiente aberto, como um lago, rio ou até mesmo em uma geleira.
Acredito que a instituição possui estrutura e pessoas competentes para prosseguir com o projeto.
Os códigos desenvolvidos no trabalho, tutoriais e o próprio texto  estarão disponíveis no repositório do \textit{Github}\footnote{Link para acesso ao repositório: \url{https://github.com/brunohjs/3DReconstruction}.}.


\section{Trabalhos Futuros}
\label{sec:trabalhos_futuros}

Alguns tópicos não foram considerados ou podem ser aperfeiçoados para realizar em futuros trabalhos. Entre os tópicos estão:

\vspace{0,5em}
\begin{itemize}
    \item Aperfeiçoamento nos algoritmos de filtragem e reconstrução;
    \item Triangulação direta no espaço 3D de superfícies;
    \item Fazer a reconstrução de dados coletados em um local aberto;
    \item Fazer a coleta em cenário mais complexos em simulação.
    \item Utilizar outros modelos de sonar para fazer a coleta;
    \item Utilizar a aplicação desenvolvida para fazer a reconstrução de dados coletados por outros sensores, como sensores ópticos;
\end{itemize}